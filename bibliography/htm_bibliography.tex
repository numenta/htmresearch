\documentclass{article} % For LaTeX2e
\usepackage{nips13submit_e,times}
\usepackage{hyperref}
\usepackage{url}
\usepackage{amsmath}
\usepackage{tocloft}
%\documentstyle[nips13submit_09,times,art10]{article} % For LaTeX 2.09


\title{Annotated Bibliography For HTM Researchers}


\author{
Subutai Ahmad \thanks{Send all flames and blame to him.} \\
Numenta, Inc.\\
Redwood City, CA 94063 \\
\texttt{sahmad@numenta.com} \\
\And
Yuwei Cui \\
Numenta, Inc.\\
Redwood City, CA 94063 \\
\texttt{ycui@numenta.com} \\
}
\newcommand{\fix}{\marginpar{FIX}}
\newcommand{\new}{\marginpar{NEW}}

\nipsfinalcopy % Uncomment for camera-ready version

% Dots for TOC
\renewcommand{\cftsecleader}{\cftdotfill{\cftdotsep}}

\begin{document}


\maketitle

\begin{abstract}
This document contains an annotated bibliography targeted towards those doing
active research on Hierarchical Temporal Memory (HTM) particularly the class of 
cortical algorithms initially defined in \cite{Hawkins2011}. The focus is on
neuroscience, though there are occasional other references. The document consists
of (possibly very brief) descriptions of specific papers. These descriptions
emphasize the relationship to HTM Theory and don't necessarily summarize the
paper itself. There is typically exactly one paper, or a couple of tightly
related papers, for each subsection. We hope the content is useful for finding
appropriate citations when writing HTM related papers. The material
could also be useful in understanding the neuroscience behind HTM theory in more
depth. One very useful feature is that the references section includes the full
abstract of each paper. The bibtex file for these references are exported from
the public Mendeley HTM Neuroscience group.
 \\

\emph{Date: \today}
\end{abstract}

\clearpage


\renewcommand{\contentsname}{Table of Contents}
\setcounter{tocdepth}{2}
\tableofcontents

\clearpage

\section{Neuroanatomy}

As HTM is a theory of the neocortex, we include review papers on architecture
of the mammalian cortex, including the laminar and columnar structures of the
cortex and thalamacortical circuits.

\subsection{Thomson: Cortical anatomy}

These papers by Thomson \cite{Thomson2003,Thomson2007,Thomson2010} are dense
but contain a lot of detailed information about the connections into, out of,
and within the various cortical layers and between different types of neurons in the cortex.

\subsection{Buxhoeveden: minicolumn review}

\cite{Buxhoeveden2002} Note from Jeff: \emph{this is the best review article I
know about mini-columns. Start here.}

The article also contains a bunch of data on typical measurements. From this you
can gather that minicolumn widths vary quite significantly between species.  The
width can range from 20-60 microns, with about 40-80 microns distance in between
(center to center).  So there are 12 to 25 minicolumns per mm of cortex.

\subsection{Constantinople and Bruno: parallel systems in the cortex}
This recent paper by Constantinople and Bruno \cite{Constantinople2013}
challenged the  classical belief of sensory processing pathway along
$L4\rightarrow L2/3 \rightarrow L5/6$ among cortical layers.  Instead, it
provided new evidence supporting the idea that superficial layers
($L4\rightarrow L2/3$) and  deeper layers (L5/6) act as parallel systems.

\subsection{Horton: minicolumn naysayer}

This paper \cite{Horton2005} is a good summary of the arguments against the
importance of columns. We should point out that the term "column" is very
ambiguous - this article reviews all the variations but is also a bit muddled
about them. In HTM the term  refers strictly to mini-columns. Phenomena like
macro columns, hyper columns, ocular dominance columns, etc. are emergent
phenomena that group together lots of mini columns. They probably arise as a
function of learning, and not really required in a strict sense. You can get
lots of variations in these macro-column structures and that would be fine. So
most of the article does not contradict HTM theory.

Horton completely disagrees  with the notion  that minicolumns have
any functional significant whatsoever.  He states, "dubious is the concept that
minicolumns are basic modular units of the adult cortex, rather than simply
remnants of foetal development".

Again we need to be precise about language. Most people debate whether the
mini-column is THE  basic unit of cortical function, as if the connections
between cells in a mini-column compute something.  In HTM theory we don’t make
this claim, it’s false, and it leads to a lot of confusion.  The cellular layer,
not the mini-column, is a unit of processing. All we claim is that cells in a
mini-column share similar feedforward response properties and that they mutually
inhibit each other.  We don’t require this inhibition be local to just the
mini-column, in fact it may very well extend to surrounding columns.  Horton
admits that mini-columns exist physically (pretty much everyone does), but he
argues that they aren’t functionally important.

One of his arguments is that there is a profuse connectivity across columns.
This argues against the idea that the mini-column is the unit of computation but
it is what we need in HTM layers.

The only claim really under contention, vis-à-vis the Horton paper, is whether
cells in a mini-column share feedforward receptive fields.  As the paper points
out it is extremely difficult to detect this, as mini-columns are very narrow
(35-50um wide and 2500um tall).  It is also likely that the cells in a
mini-column are not exactly linear.  They can move about as much as they want as
long as the connections are maintained. Neocortical tissue is more like tapioca
and less like wood.

We propose that cells in a mini-column will exhibit very different responses
during learned sequences.  This exact behavior has been documented.  However,
some people argue this proves that the cells in the mini-column do not share
common receptive fields!  So you have to be careful when reading claims.  Was
the animal awake, was it a brain slice, what exact stimulus is being used?

HTM theory provides an elegant explanation for the conflicting data and
confusion about mini-columns.

\section{Sparse Distributed Representations}

A core aspect of HTM theory is that sparse distributed representations (SDRs)
are used everywhere in the brain. In this section we summarize neuroscience
research papers supporting various aspects of this idea.

\subsection{Olshausen: Sparse coding}

The 1996, 1997 papers \cite{Olshausen1996, Olshausen1997} by Olshausen and Field
represent the first computational papers on sparse representations in the
cortex. Their work has been very influential in the machine learning and
neuroscience. The 2004 paper \cite{Olshausen2004} is shorter and easier to read,
more of a review.

\subsection{SDR object representations in IT}

\cite{Kiani2007} is an extensive study of SDR-like properties in IT. They
measured responses of 600 neurons in monkey IT to 1000 images. They analyzed the
similarity between all pairs of responses. By doing agglomerative clustering
they show that objects tend to cluster according to natural categories. i.e.
similar objects have similar representations. They use many different distance
methods, though only one is really shown in detail.

The selectivity of single cells was imperfect compared to cell population as a
whole. Many cells discriminated between combinations of categories. In other
words it is a distributed representation.

They showed that lower level simple and complex features (roughly equivalent to
V1, V2) cannot account for these similarity metrics.

Note that each image was only presented for 105ms, so it's essentially a flash
inference case.

\subsection{SDR episodic memory in hippocampus} This paper \cite{Wixted2014}
provides supporting evidence that a sparse distributed neural code  is used for
episodic memory in human hippocampus. Specifically, they showed that  (i) a
small  percentage of recorded neurons responded to any one target and (ii) a
small percentage of targets  elicited a strong response in any one neuron

\subsection{Babadi: Sparseness and Expansion}


The paper \cite{Babadi2014} is tangentially related to our SDR work. I am
including it here only because we might want to cite it in our
theory papers. I think there are several key differences:

\begin{itemize}
\item They are focused on sparseness combined with expansion. This is a bit
strange, since expansion is not required and not found everywhere in the cortex.

\item They require full connectivity matrix between the input layer and the
encoding layer

\item They are using analog weights and a very different distance function
(Euclidean distance instead of overlap)

\end{itemize}

\section{Pyramidal Neurons}

Pyramidal neurons are the most numerous excitatory cells in the cortex and play
important roles in advanced cognitive functions. Understanding the
structure and function of
these neurons is important to HTM research.   This section  contains research
and review papers on properties of pyramidal neurons.

\subsection{Spruston: Pyramidal neuron review}

This paper \cite{Spruston2008} contains a very nice review of
pyramidal neurons, including their various regions, active dendritic
properties, number of synapses, etc. They focus on the commonalities across
different cortical areas and hippocampus.

\subsection{Polsky: computational subunits in dendrites of pyramidal cells}
Polsky et al \cite{Polsky2004} showed that the thin basal dendrites of pyramidal
cells provide a layer of independent computational 'subunits' that sigmoidally 
modulate their inputs prior to global summation. This challenges the often reported 
global linear or sublinear summation of synaptic inputs in pyramidal cells.

Similar ideas have been proposed in an earlier paper \cite{Poirazi2003}, which
shows that pyramidal cells is best modeled as a "two-layer" neural network. 

\subsection{Rah: Thalamocortical inputs targeted proximal basal dendrites}
Rah et al \cite{Rah2013} used array tomography, a high resolution optical 
microscopy method to examine the spatial distributions of thalamocortical (TC) 
inputs onto L4 and L5 pyramidal neurons. 

\emph{Main results}:
\begin{enumerate}
\item TC synapses primarily target basal dendrites in L5 and proximal apical
dendrites in L4 (Fig. 8).
\item TC inputs are biased toward certain branches and, within branches,
synapses show significant clustering with an excess of TC synapse nearest 
neighbors within 5-15 $\mu m$ compared to a random distribution
\end{enumerate}

\subsection{Yoshimura: vertical and horizontal inputs to L2/3 neurons}

Yoshimura et al. \cite{Yoshimura2000} examined properties of horizontal inputs 
(lateral connections) and vertical inputs (feedforward inputs from L4) to L2/3 
pyramidal cells. They found that vertical inputs (L4->L2/3) evoke large amplitude 
EPSPs of L2/3 pyramidal neurons, whereas horizontal inputs 
(lateral inputs from other L2/3 neurons) caused small amplitude EPSPs with 
large variation.

\section{Active Dendrites}

An important function of the neocortex is to generate predictions of what will
happen in the future. In HTM theory proximal and distal dendrites recognize
patterns and depolarize cells into a "predictive state". Over the last 20 years,
the properties of active dendrites have become a topic of great  interest to the
neuroscience community. Active dendrites seem to be coicidence detectors,
detecting a temporal coincidence of activity within a portion of the dendrite
through "NMDA spikes". These spikes usually don't cause a neuron to spike but
they  seem to prime the cell and make it more likely to generate an axon
potential, and generate it faster than it would have otherwise.

HTM theory says that these properties form the underlying neural mechanisms for
prediction and inference. This section contains numerous papers describing
current neuroscience knowledge on active dendrites.  Some of the papers also
show that apical dendrites also have very interesting special properties.  Much
of our current HTM research is focused on understanding dendritic properties in
detail.

\subsection{Antic: Decade of NMDA Spikes}
From Jeff: \emph{This is a short and relatively easy to read paper that covers
 the basics of NMDA dendritic spikes including some of the history.  It might be a
good introduction to the topic.}

The paper \cite{Antic2010} describes evidence for the HTM "predictive state". A
non-apical NMDA spike can depolarize a cell body (page 2998, middle of right
column). The time course of this depolarization is interesting - the paper shows
evidence for a more sustained depolarization than we typically model in HTMs.
This could be used to help combat temporal noise or it could support the "learn
on one cell" mode.

They also provide evidence that NMDA spikes are highly localized events
incorporating a small dendritic segment. They are localized in space (10-40
microns) and localized in time. See page 2998, middle of left column.

\subsection{Larkum: Synaptic integration in tuft dendrites}

Although this paper \cite{Larkum2009} is primarily about apical tuft dendrites
it is also a good introduction to dendritic spikes in general, including basal
dendrites.  Figure 4H is a good summary diagram of how to think about a typical
pyramidal neuron where the apical tuft dendrites act as a set of coincidence
detectors similar to how the basal dendrites act as a set of coincidence
detectors.  The apical tuft dendrites generate a Ca spike and the basal
dendrites generate a somatic Na action potential.

\subsection{Branco: How dendritic segments integrate their inputs}

This paper \cite{Branco2011} contains some useful data on dendritic segments
and how they integrate their inputs. Summary of findings:

\begin{enumerate}

\item Proximal dendrites sum over their synapses linearly. Synapses sum linearly
and inputs must converge at the exact same time. They also show that more distal
dendritic branches show threshold like (sigmoidal) response properties. This is
nice confirmation for our model.

\item Distal dendrites have broader temporal integration windows. Quote: "We show
that single cortical pyramidal cell dendrites exhibit a gradient of temporal
summation and input gain that increases from proximal to distal locations. This
suggests a progressive shift of computational strategies for synaptic inputs
along single dendrites.

Near the base if inputs are more than a few msecs apart, the evoked potential
drops significantly. Confirmation that proximal dendrites act as coincidence
detectors. Segments farther away from the base can sum over longer temporal
intervals. At the most distal locations, evoked potential stays constant for
much longer than 10 msecs.

Another way to phrase this is that proximal segments require precise synchrony
(coincidence detection), distal segments do not require such precise timing.


\item More distal segments have higher gain (sigmoid gain) and require fewer
synapses to fire. As a consequence, they also state that really distal segments
can be as effective or even more effective than proximal dendrites in driving
axonal output. Not sure what to make of this.


\item They show all this in Layer 2/3 as well as Layer 5 pyramidal cells.

\end{enumerate}

Some reactions to temporal property: We don't model the timing property of
distal segments today, but this property could be very useful. For example
longer temporal scales can be helpful in dealing with temporal noise. A longer
scale will make us more resistant to temporal insertions. It could also be used
to create connections from a single segment to time steps t-1 and t-2. This will
make us more robust to temporal deletions. (This is similar to our old
pooling idea.)


Distal inputs into L4 can be coming in at different time scales (motor commands
could be slower than sensory input). A longer temporal integration period may be
necessary for proper sensorimotor inference.

If apical dendrites have much larger temporal integration windows, this could be
very useful for feedback which necessarily has a slower time scale.

\subsection{Losonczy: Integrative Properties of Active dendrites}

This paper \cite{Losonczy2006} contains a detailed study of exactly how many
synapses are required to initiate a dendritic spike in hippocampal pyramidal
neurons.

Their study suggests that 17-20 active synapses are required for
dendritic spiking. They show that the synapses need to be within 6 msecs,
confirming the hypothesis that these dendrites are acting like a coincidence
detector.  If you spread out the activity to 50 msecs you do not see spiking.

The spiking is specific to a branch. The synapses do not need to be
tightly spatially clustered but can be anywhere on a branch. They quote that
a branch typically has 300 to 400 synapses. This suggests that a branch can
detect noisy patterns (the threshold is lower than the total number of cells)
or that they are detecting a union of patterns, or both.

\subsection{Major: Active properties of dendrites}

This paper \cite{Major2013} cites evidence that distal dendritic spikes often
have a weaker modulatory effect. (Proximal connections are thought to have a
stronger effect.)

They review different types of cell models.

They provide numbers regarding the minimum number of synapses required to
trigger dendritic spikes that match up very well with our formal analysis of
SDRs.  NMDA spike in distal dendrites can be evoked by as few as 10 active
synapses that are clustered together (page 17, top left) or 20 spread along
the branch. Note that the effect of spatial clustering is slightly different
than reported in \cite{Losonczy2006}.  However as they point out,
these numbers are rough and can vary dynamically (e.g. as a result of
dendritic inhibitory input).

\subsection{Major: NMDA spikes in basal dendrites of pyramidal neurons}

This paper \cite{Major2008} shows that glutamatergic inputs clustered over 
approximately 20-40 microm can elicit local NMDA spike/plateau potentials in 
terminal basal dendrites of cortical pyramidal neurons.

\subsection{Golding: calcium spike in apical dendrites of pyramidal neurons}

This paper \cite{Golding1999} shows calcium spikes are initiated in the apical 
dendrites of CA1 pyramidal neurons and drive bursts of sodium-dependent action 
potentials at the soma. A predicted cell could fire bursts of action potentials 
through this mechanism. 

\subsection{Larkum: BAC firing hypothesis}

Apical dendrites seem to have a special integration zone which can cause that
region of the cell to be depolarized for sustained periods of time.

In addition, feedback input to apical dendrites by themselves can cause strong
action potentials.   If this is coincident with matching feed forward input, the
cell bursting can be sustained for a period of time. Mechanisms for that are
reviewed. Figure 3a is an example of this.

\emph{
...this article \cite{Larkum2013} has dealt with the biophysical
evidence for the existence of an associative firing mechanism in pyramidal
neurons and its influence on the input/ output function. This degree of
integration between the micro- and macroarchitecture, as well as inbuilt
complexity at the cellular level, invites speculation about whether and how the
whole system utilizes this feature. The importance of this mechanism
conceptually is that the pyramidal neuron is able to detect coincident input to
proximal and distal dendritic regions, investing the cortex with an inbuilt
associative mechanism at the cellular level for combining feed-forward and
feedback information.}

\emph{The BAC firing hypothesis presented here offers a cellular mechanism that
addresses a number of questions about the cortex. It suggests that the
pyramidal neuron cell type is an associative element which carries out the same
essential task at all cortical stages: that of coupling feed-forward and
feedback information at the cellular level.}


Martinotti neurons blocking dendritic activity - what is this for? Need to
review. Could this turn off pooling when columns burst?


\subsection{Palmer: NMDA Spikes in L2/3 neurons}

Few papers talk about dendritic NMDA spikes in Layer 2/3 pyramidal neurons.
That is the topic of this paper \cite{Palmer2014} showing that Layer 2/3 cells
act similarly to the more studied L5 and Hippocampal Ca1 pyramidal neurons.

Another study \cite{Larkum2007} studies dendritic spike in the apical dendrites of L2/3 pyramidal neurons. They conclude that L2/3 neurons can generate dendritic spikes similar to L5 neurons.

\subsection{Yuste et al: Apical Dendrite as a segregated functional compartment}

Yuste et al \cite{Yuste1994} proposed that the apical dendrite and basal dendrite could act as two segregated functional compartments. The authors showed that the apical zone can be activated (generated Ca2+ spikes) even when Na and K conductances are blocked. 

In a more recent paper, Larkum et al \cite{Larkum2009} made a similar proposal (Fig. 4) with more advanced experimental techniques. They further showed that a few NMDA spikes were needed to trigger Ca2+ spikes in the apical zone.


\section{Synapses and Plasticity}

Learning requires changes of synaptic connections between neurons. In this
section we review papers on synaptic plasticity, with a focus on
experience-dependent formation and elimination of synapses.

\subsection{Hebb and his learning rule}

Although this is often stated as Hebbian learning, Hebb did not in fact write
"Neurons that fire together wire together". Here is what he actually wrote in
\cite{Hebb1949}:

\begin{quote}
\emph{Let us assume that the persistence or repetition of a reverberatory
activity (or "trace") tends to induce lasting cellular changes that add to its
stability.… When an axon of cell A is near enough to excite a cell B and
repeatedly or persistently takes part in firing it, some growth process or
metabolic change takes place in one or both cells such that A's efficiency, as
one of the cells firing B, is increased.}
\end{quote}

The wording of his actual statement is really interesting. First, he
talks about growth processes and metabolic changes. This is consistent with the
notion of synapses actually growing.  Second he talks about axon of cell A being
near cell B - this is consistent with notion of potential synapses.  Third he
talks about cell A persistently taking part in cell B’s activity. This implies a
temporal and causal relationship, not just instantaneous correlation. In fact
he discusses sequences quite a bit in his book. If  you  really  want to read
into it, his exact wording can be seen to be consistent with active dendrites
and the HTM learning rules.

\subsection{Chklovskii: Potential synapses}

This paper \cite{Chklovskii2004} includes evidence and arguments for dynamic
synapse formation,  potential synapses, etc. Jeff says on the NuPIC Theory
mailing list:

\emph{Classic Hebbian learning is about strengthening synapses, but we are
suggesting that new synapses are formed. I think you are asking how this is
possible biologically. The idea for “potential synapses” comes from a researcher
Chklovskii.  An axon and dendrite that are near each other but that are not
connected can still sense pre- and post-syaptic activity and this is sufficient
for them to start to grow a new synapse when they are active at the same time.
What is actually is grown is a “spine” that connects the axon and dendrite.  The
synapse is on the end of the spine.  In addition, glial cells have been shown to
act as intermediaries.  They can encourage a dendrite and axon to move/grow
closer together when they fire at the same time.  Glial cells effectively
increase the number of potential synapses.  Finally, the ends of the dendritic
and axonal tress are constantly growing I different directions trying to find
new useful connections.  It is now well known that synapses and spines form
during learning and they also disappear.  This can happen rapidly, in a matter
of minutes in some cases.}

\emph{I believe the concept of potential synapses and the growth of new synapses
applies to both proximal and distal synapses.  In our temporal memory
implementation we decided to implement it differently purely to make the
software more efficient and to use less memory.  Instead of maintaining a large
pool of potential synapses, most of which would have 0 permanence, we choose
from the set of active cells.}

\subsection{Experience-dependent synaptogenesis}

The review paper by Bailey and Kandel \cite{Bailey1993} presents numerous early
researches supporting synapse formation during formation of long-term memory.

Zito and Svoboda 2002 \cite{Zito2002} is a short review of activity-dependent
synaptogenesis in the adult cortex. This review first presented the common view
of synaptic plasticity, which argues initial establishment of synaptic
connections occurs independent of learning, and experience refines existing
synaptic connections, rather than create new ones (para 2, page 1). Zito and
Svoboda then presented a different view, which argues "short-term plasticity
depends on existing synapses, but long term changes in synaptic strength are
accompanied by structural rearrangements, through formation or elimination of
synapses. This second view is consistent with the current HTM algorithms.

A more recent review paper by Holtmaat and Svoboda \cite{Holtmaat2009} presents
more recent findings with imaging techniques. I find the numbers in
supplementary table S1 very useful, it shows what fraction of synapses are
stable from many different studies.

\subsection{Losonczy: Branch specific plasticity}

HTM theory requires that if a dendritic segment becomes active and then that
cell later fires, that we learn on that specific segment. This requires a
coupling between segment activity and later activity at the soma. This is a
fairly strong and specific prediction of HTM learning rules.

This paper\cite{Losonczy2008} seems to be the first demonstrated evidence for
this type of plasticity rule. First they show that there is branch (segment)
specific potentiation (learning), isolated from other branches. Second, they
show that the branch specific potentiation requires a back action potential
(top left of page 439).

They also show that there are weak branches and strong branches.  The strong
branches tend to be near the soma (proximal) and the weaker branches tended to
be terminal (distal). They also show that eventually the weaker branches can
cause the cell to fire – this is not currently modeled in HTM.

Figure 1 in their supplementary materials is very closely related to HTM
theory. They hypothesize that dendritic segments are used for storing
individual elements of sequence and that the efficiency of processing
sequences will be increased. To quote: \emph{The repeated occurrence of the
sequence item [snip]will be stored in the increased branch coupling strength,
such that subsequent arrival of the item will have an enhanced efficacy and
improved capability to evoke a precise output.} Although they don't explicitly
discuss prediction and high order sequences, the experiments and results in this
paper are very consistent with HTMs.


\subsection{Yang: Branch specific dendritic learning}

This amazing paper \cite{Yang2014} studies learning on specific dendritic
branches in awake rats while they go through normal behavior. It shows learning
induced dendritic branch specific formation of synapses, which is consistent
with the temporal memory algorithm and high order sequences. This paper deals
with high order motor sequences. Ignore the sleep part.


\subsection{Imaging the growth of synapses}

Most recent studies are use fancy image techniques to study synapse formation.
Trachtenberg et al \cite{Trachtenberg2002} first shows experience-dependent
synapse formation and elimination with long-term imaging. Niell et al.
\cite{Niell2004} contains a movie showing how fast synapses and dendritic
segments can change dramatically within 24 hours.

\section{Sequence Learning}

In this section we review evidence of sequence learning in the cortex. One
prediction of HTM theory is that unexpected input will cause  column-level
bursting  activity in the cortex whereas predicted inputs are represented by
SDRs. Expected input will lead to extremely sparse activity (one or two cells
per active column). Another prediction is that formation and elimination of
lateral connections between neurons is a key mechanism for sequence
learning. These predictions are supported by the following studies.

\subsection{Vinje and Gallant: Temporal sequences are sparser}

Part of a classic set of studies for HTM sequence learning. This paper
\cite{Vinje2002}
demonstrates increasingly sparse neural activity when a neuron gets larger
spatio-temporal context compared to flash input. Moreover the same set of
neurons tend to fire given the same temporal context. (They show temporal
context by creating� that simulate eye saccades.)

There are some problems with the study. For one, the eye movements are not
initiated by the monkey. Instead the movie simulates what the monkey might see
during saccades. This is a huge issue. Their analysis also assumes there are no
temporal correlations in neural activity. Another big problem since temporal
pooling creates temporal correlations. They don't really think in terms of the
monkey learning sequences. Etc.

\subsection{Milier: Visual stimuli recruit intrinsically generated cortical ensembles}

This paper \cite{Miller2014} provides some supporting evidence of the Temporal Memory algorithm. 

Neuroscientists traditionally think that neurons in the cortical encode sensory
stimulus individually: each neuron has some preferred stimulus, which can be
measured through receptive field (RF) mapping. When a stimulus comes in, a
neuron will fire if it matches its RF well, and will stay silent otherwise.

What's interesting in this paper is that the authors suggest groups (ensembles)
of neurons , rather than individual neurons, are the functional units of
cortical activity. When a stimulus is presented, cortical activity is dominated
by coactive groups of neurons.  Presumably these ensembles are supported via
learned recurrent excitatory connections, since they also appears during
spontaneous activity.

I find this consistent with the temporal memory algorithm (previously known as
the CLA algorithm), which suggest external stimulus triggers a unique sequence
of cells, which are connected via lateral connections. However, the technology
used in this paper (two-photon imaging) has a relatively poor temporal
resolution, so it is not possible to tell whether the "coactive group of
neurons" fires sequentially or simultaneously.
Chetan:What else is interesting is that they found single neurons to participate
in multiple ensembles, regardless of how specialized they were for a stimulus at
an individual level. (See section titled "Single Neurons Participate
Promiscuously in Multiple Ensembles").

Also, this is interesting: "Taken together, our findings demonstrate that when
individual neurons are activated, they are more likely to be activated together
with a specific set of other neurons as an ensemble. At the same time,
individual neurons can participate in multiple ensembles, dynamically
reorganizing their allegiance with different sets of neurons."

\subsection{Meyer: surprise responses in IT}

Subutai: I've been trying to locate strong experimental evidence that surprise
inputs result in cells in mini column firing (aka bursting).  The attached paper
doesn't quite show that, but it has some other relevant information
\cite{Meyer2011}.  Unpredicted stimuli elicits a stronger response in IT than
predicted stimuli. The mean firing rate of neurons is higher with unpredicted
stimuli (Figure 2). In figure 2A you do see a slightly increased mean response
rate to predicted stimuli (above baseline) but the average response is higher
for unpredicted stimuli. All this is consistent with column bursting in temporal
memory. In this paper you can't tell whether the surprise is along a minicolumn
or not. Additional experiments showed that unpredicted firing had slightly
longer latency than predicted firing. This is consistent with predicted cells
firing earlier due to depolarization. Predicted firing was about 5 msecs faster
than unpredicted firing, which is also about what you would expect.

\subsection{Fishman: Oddball responses in auditory cortex}
 
From EEG recordings, people know that unexpected stimulus ("deviant" sounds in
this study), will elicit stronger response. This paper \cite{Fishman2012}
studies the neural mechanisms and brain regions underlying this effect in the
primary auditory cortex. They confirmed that spiking response were larger when
elicited by the unexpected stimulus, and the difference between expected and
unexpected stimulus were more prominent in later activity. The unexpected
stimulus can be introduced either in a oddball paradigm in which rare deviant
tones are randomly interspersed among frequent standard tones (Fig. 1B), or in
a random sequence where tone is unexpected (Fig. 1C).

One downside of this study is they used a very simple sequence for the oddball
paradigm. It will be more interesting to see whether the same phenomenon holds
when they play a more complex tone sequence (an oddball within a melody). They
conclude that the difference between unexpected stimulus and repeated expected
stimulus is due to stimulus specific adaptation (rather than prediction).

\subsection{Gavornik: sequence learning in V1}

\emph{Main result}: Gavornik and Bear \cite{Gavornik2014} shows that neurons in
V1 adapt to sequences. Apparently sequence learning has not been demonstrated in
V1 before. Mice were trained on a particular sequence of sinusoidal gratings,
ABCD.  After training ABCD evoked a much larger response than DCBA. Also much
larger than ABCD in a control group.  They found that timing makes a difference.
A change in the timing of ABCD causes lower responses as well.

\emph{Other results}: The mice learn very quickly, within one day (2 mins of
total
visual stimulation). During this time they saw the sequence 200 times.  The
sequences were robust to dropout. ABCD had similar responses to A\_CD, both of
which were larger than E\_CD.  Their results don't seem to be layer specific.
They find larger responses to trained sequences in all layers.   Sequence
learning was stimulus specific and localized; it did not transfer to the other
eye.

\emph{Oddity 1}: Why would responses be larger for a trained sequence? We would
expect
sparser activity after training. They are measuring evoked field potential in
layer 4. After discussing with Paxon it seems there is some controversy as to
what LFP or VEP actually measures. It measures activity in all cells: both
inhibitory and excitatory but there is evidence that it might be more sensitive
to inhibitory cells.  In either case a larger response could be consistent with
HTM theory, since there would be a fair amount of inhibitory suppression within
mini columns for predicted sequences.

\emph{Oddity 2}: they claim that sequence learning occurs without NMDA receptor
activation (page 3). I didn't understand that portion. Does that mean NMDA
spikes are not required?

\subsection{Martin: Functional heterogeneity in neighboring neurons of V1}
 
\emph{Main result}: Although neurons in V1 are clustered according to their stimulus
 preferences, responses to complex visual stimuli are highly heterogeneous between 
 adjacent neurons \cite{Martin2013}. In response to gratings and visual noise, signal and noise 
 correlations were well correlated with each other, but less so for responses to movies. 
 
 Similar results are reported in \cite{Yen2007}, which shows that when stimulated with 
 complex, time-varying natural scenes (i.e., movies), nearby striate neurons exhibit highly 
 sparse and heterogenous responses.
 
\emph{Other result}: "trial-to-trial fluctuations (termed �noise�) were poorly correlated between 
 neighboring neurons, suggesting low degrees of common input." This is consistent with 
 our assumption that cells in the same mini-column has different lateral connections.
 
\section{Temporal Pooling and Invariances}

The idea of temporal pooling is that after learning, predicted inputs will
lead to stable (invariant) representations in higher levels of the cortical
hierarchy despite changes of inputs at lower levels of the hierarchy. We
summarize supporting neuroscience evidences for this idea below.

\subsection{Li and DiCarlo: Learning invariances}

This paper \cite{Li2008} very closely supports HTM temporal pooling ideas,
including the role of sensorimotor inference in forming invariant
representations.  A detailed summary follows:

They found neurons in monkey IT that responded strongly to a particular object
P (preferred) and moderately to another N (non-preferred). Tested position
tolerance of these objects to get a baseline.  The difference between the
responses to P and N is the selectivity for this neuron.  They did this
for about 100 neurons in two monkeys, and many different objects.

They then altered temporal contiguity of stimuli and tested the effects on these
invariant representations.



In the experiment, monkeys viewed altered stimuli for several hours. Basically
there were three positions for each object: centered, 3 above and below. For a
given P object, the position above or below was designated the swap position.
As eyes moved around, a P object would be shown in the swap position but was was
replaced by the N object during saccade. (Each saccade lasts 23 msecs.) This
would happen every time P object was shown in the swap position.  A P object in
the non-swap position didn't change when the money saccaded. (The swap position
was randomly changed for different objects.)  The monkey is effectively blind
during a saccade, so they don't see the change - they just see the new object
after fixation.

Their prediction: invariance selectivity at swap position would change.
Selectivity in the non-swap position won't change.  This is what happened. At
swap position cells did not discriminate as well between P and N. Neuron became
more responsive to N and less responsive to P, but only at that location. Change
was location specific and shape specific, so very specific to this particular
invariance. These changes cannot be explained by attention affects or
retinotopic adaptation (everything was counter balanced).

However another paper of theirs seemed to indicate that temporal contiguity
alone was not enough. Looks like eye movements may also be required (this is
their reference 10 - Cox and DiCarlo, 2005. Attached paper 2).

They call this effect �Unsupervised temporal tolerance learning� or UTL. This is
just temporal pooling for us.  The behavior they describe is exactly the pooling
behavior we would expect in layer 3.

In \cite{Cox2005} they have the same experiment as \cite{Li2008}, but
here they used human subjects and a same-difference psychological test.  
Swapped objects were much more likely to be confused than non-swapped objects.
They did the same experiment without eye saccades, and did not see the effect. 

\emph{"Moreover, the confusions are predictable in that they are what is
expected if the visual system assumes that object identity is stable across the short time
interval of a saccade."}


\subsection{Isik et al: Learning invariance using temporal associations}

Yuwei: Here \cite{Isik2012} is a modeling study from Tomaso Poggio's lab that
 is very
relevant to the idea of temporal pooling. This paper aims to explain the
"invariance disruption" experiments, which is the study Subutai found that
supports temporal pooling. In that experiment, Li and DiCarlo
\cite{Li2008} showed that
individual IT neurons change their selectivity in a position-dependent manner
after exposure to the altered visual experiment. The modeling study here used a
temporal association learning rule to learn transformation invariance through
natural visual experience. The basic underlying idea is very similar to temporal
pooling: since the external visual scene usually changes at a slow time scale,
"temporal adjacency" is a good cue that two images are of the same object.

I think the major difference between this study and what we have is they don't
distinguish between predicted and unpredicted inputs. In our algorithm, we only
start temporal pooling when the input is predicted. They also don't have a motor
component in their model. Nevertheless, I feel this line of study (developing
invariance with temporal association rules) is very relevant for us to think
about temporal pooling. i am planning to take a look at some papers along these
lines.

Jeff:  In addition to Yuwei's observations I would add:  - They make a
distinction between generic transformations� (such as image translations) and
�class-specific translations� (such as rotation through plane).  They argue that
generic transformations apply to all objects but class-specific transformations
do not.  I am struggling to see if this distinction really exists.  I would
prefer that temporal pooling works the same in all cases and that this
distinction doesn't exist.


Another thing they talk about which I found odd is they ask, why should
training continue throughout life?  They say generic transformations can be set
early in life.  This seems ridiculous to me.  Perhaps they are trying to show
that HMAX (which doesn't learn at all) is a good model.  They show a chart
(figure 2) showing that for translation fixed HMAX is as good as learning
temporal pooling.

\section{Feedback}

What is feedback? Information processing in the cortex is hierarchical and
bidirectional. Sensory input to the hierarchy occurs at the lowest levels and
flows from the bottom to the top, i.e. in the feedforward direction. We would
characterize this process as bottom-up. We define lateral connections as those
connections that occur between neurons in the same \emph{layer} (not level). Here
we define feedback as none of those. Specifically here they are projections from
higher levels back to down to lower levels. We do not include
lateral connections in our definition of feedback (this distinction is often
confusing in the literature).

We leave open for now other types of connections, such as cortico-thalamic and
intralaminar.  Some might say that cortico-thalamic pathways are involved in
attention, which is a top-down process and hence involves feedback. This would
not be unreasonable. Similarly there might exist intralaminar connections
(within a level) to apical dendrites where more stable patterns bias less stable
patterns. This could also be reasonably called "feedback". We will place these
categories of connections firmly in our "gray area" bag until we better
understand the precise mechanisms and anatomy.

\subsection{Makino: learning enhances impact of top-down pathways}

Makino's paper \cite{Makino2015} is an excellent study of learning in feedback
pathways. The paper studies activity and learning in layers 2/3, 4, and feedback
projections into layer 1 from a higher region. Their experimental protocol
involved active visually guided avoidance and as a control, passive visual
experience. They examined plasticty in area V1 in mice. They identified a
specific higher level (RSC) that projects to V1 as a major source of top-down
input, and restricted top-down measurements to axons projecting from RSC to V1.

They found that during active learning top down activity increased over several
days. They measured  top-down activity as  the activity of axonal boutons
projecting from RSC to layer I. This increase did not occur for the passive
learning case. Activity in Layer 2/3 and L4 became much sparser in general. The
difference in Layer 2/3 was larger in the active learning case than in the
passive learning case.

Activity in Layer 2/3 neurons and RSC boutons ramped up significantly during the
active trials, in anticipation of specific visual events.  This did not occur
in the non-active condition.  Layer 4 did not show any such ramp up.  They show
that RSC activity is required for the Layer 2/3 ramp up. RSC inactivation also
impaired the performance of the active learning task.

They also measured activity of interneurons and suggested that top-down activity
in general inhibits bottom-up pathways. As they state, \emph{"such learning-related
shifts in the neocortical operation mode may be a fundamental feature of
cortical computations underlying associative feature binding ... by effectively
ignoring irrelevant fluctuations in sensory inputs resulting from the noise in
the sensory environment."}



\section{Thalamocortical Pathways}

\subsection{Sherman and Guillery: The Book}

For those interested in going deeper into the role of the thalamus, this is an
excellent book \cite{Sherman}. Suggested by Jeff, it is a well written summary
of a modern view of cortico-thalamic connections. It describes, for example, the
connections between every cortical region and the thalamus including the role of
sub-cortical motor centers. It does require some neuroscience background but is
much easier to read than many of the really dense neuroscience papers. The
diagrams are also very clear.

\section{Sensorimotor Inference}

The idea of sensorimotor inference is that cortex makes predictions of future
inputs based on both the current inputs and a copy of motor command that is
about to be executed.  Some supporting evidences is summarized below.

\subsection{Bartoli: SDR representations of tools}

This paper \cite{Bartoli2014} is concerned with merging of information from
visual properties of tools to the motor coordinates for using and interacting
with those tools. Looking at premotor and motor populations of neurons coding
for specific hand configurations.

An "affordance" is the set of stuff that can be done to an object.  It is the
possibility of some actions that can be performed on an object.

Neurons in premotor cortex code for object specific and grip specific actions.
(Rochat, 2010). A subset of these neurons fire for executing a specific
hand-object and a specific visual representation of that object. (Murata 97)
This activity is indepedenent of whether that action actually occurs later.

This paper shows that the same specific patterns observed in monkeys also occurs
in humans.  They visually show specific tools to people, and observe very
specific "motor plans" arising as a result of the stimuli. The main contribution
is to "close the gap" between monkey and human data. In addition, they show that
very specific visuomotor representations can form in humans as soon as 150 msecs
after visual presentation of the object.  The did a TMS study, so the stimulated
premotor cortex (they did not perform direct neural recordings).  They did
record activity from hand muscles.

\subsection{Saleem: Integration of visual and locomotion in V1}

The Saleem study \cite{Saleem2013} clearly showed that V1 (at least in mouse) 
integrates visual motion and locomotion signals during active navigation. Specifically,

\begin{enumerate}
\item Many V1 neurons are tuned for running speed (Fig. 1-2). There is a specific preferred 
running speed, more than a nonspecific increase of activity level \cite{Niell2010}.
\item Nearly half of the V1 neurons were reliably driven by the combination of visual speed and 
run speed (i.e., neurons are tuned to a weighted sum of visual speed and run speed Fig. 3)
\item As a population, V1 neurons predicted a linear combination of visual and run 
speeds better than either visual or run speeds alone (Fig. 4).
\end{enumerate}

\subsection{Rauschecker: where and what pathway in auditory cortex}

The "where" and "what" pathway, which was original proposed to describe the 
dorsal and ventral pathway of the visual system,may be generalized to other sensory 
modalities. Specifically, this paper \cite{Rauschecker2000} 
showed that one region of the auditory cortex are selective to spatial locations and 
another region is selective to auditory pattern or objects. 


\subsection{Wolpert: Forward models}

This paper \cite{Wolpert1996} discussed sensorimotor inference from a
computational perspective. They proposed a forward model that uses the current
state of the motor system and motor command to predict the next state. This is
similar to our sensorimotor inference algorithms. The forward model concept in
this paper is widely used in motor control and sensorimotor inference.

\subsection{Sommer: Efference Copy}

This review paper \cite{Sommer2006} summarizes a series of studies that
established a  pathway for corollary discharge signal (the motor command copy to
sensory cortex), explains how predictive shifting of receptive field is
constructed with CD signal, and how visual stability is achieved despite
eye-movements.


\bibliographystyle{plain-annote}
%\bibliographystyle{apalike}
\bibliography{htm_bibliography}

\end{document}
